\documentclass[12pt,a4paper]{report}
\usepackage[utf8]{inputenc}
\usepackage[spanish]{babel}
\usepackage{amsmath}
\usepackage{amsfonts}
\usepackage{amssymb}
\usepackage{lmodern}
\usepackage{amsmath}
\usepackage[left=2cm,right=2cm,top=2cm,bottom=2cm]{geometry}
\usepackage{graphicx}
\providecommand{\abs}[1]{\lvert#1\rvert}
\author{Agustin Curto, agucurto95@gmail.com}
\title{Resumen de teoremas para el final \\ de Matemática Discreta  II}
\date{2016}

\begin{document}
\maketitle
\tableofcontents


\chapter{Parte A}

	\section{La complejidad de EDMONS-KARP}
	
		\textbf{\underline{Teorema:}} La complejidad de $\langle E-K \rangle$ con $n = \abs{V}$ y $m = \abs{E}$ es $\mathcal{O}(nm^{2})$.
	
		\textbf{\underline{Prueba:}} Probaremos primer un hecho por inducción para enunciarlo, sean:
		
		\begin{center}
			$f_{0}, f_{1}, f_{2}, \dotsc$ \;  la sucesión de flujos creados por $\langle E-K \rangle$.
		\end{center}
		
		El paso \textit{k} es el paso que crea $f_{k}$. Definimos funciones:
		
		\begin{itemize}
			\item $d_{k}(x) = distancia^{1}$ entre \textit{s} y x en el paso \textit{k} en caso de existir, si no $\infty$.
			\item $b_{k}(x) =$ distancia entre x y \textit{t} en el paso \textit{k} en caso de existir, si no $\infty$.
		\end{itemize}
	
		Observación:
			\begin{itemize}
				\item $d_{k}(s) = 0$
				\item $b_{k}(t) = 0$
			\end{itemize}
	
		\underline{Lema interno:} queremos probar que
		\begin{enumerate}
			\item $d_{k}(x) \leq d_{k+1}(x)$
			\item $b_{k}(x) \leq b_{k+1}(x)$
		\end{enumerate}
		
		\underline{Prueba:} lo probaremos por inducción y solo para $d_{k}$ ya que para $b_{k}$ la prueba es análoga.
		
		\begin{center}
			HI: $H(i) = \lbrace\forall_{z}: d_{k+1}(z) \leq \textit{i}, \, d_{k}(z) \leq d_{k+1}(z)  \rbrace$
		\end{center}
		
		\begin{enumerate}
			\item $H(0) = \lbrace\forall_{z}: d_{k+1}(z) \leq 0, \, d_{k}(z) \leq d_{k+1}(z)  \rbrace$
			
			Pero $d_{k+1} \leq 0 \Rightarrow $
		\end{enumerate}
		
		
	\section{Las distancias de Edmonds-Karp no disminuyen en pasos sucesivos}

	
	\section{La complejidad de DINIC}

		\underline{Teorema:} La complejidad del algoritmo de Dinic es $\mathcal{O}(n^{2}m)$.
		
		\underline{Prueba:} Como la distancia entre \textit{s} y \textit{t} en networks auxiliares consecutivos aumenta y puede ir a lo sumo entre 1 y $n-1$ entonces hay a lo sumo $\mathcal{O}(n)$ networks auxiliares.
		
		Notación: llamemos PB al proceso de hallar paso bloqueante en un network auxiliar con Dinic.

		Luego la complejidad de Dinic es $ \; \mathcal{O}(n)$ compl(PB). Para probar que la complejidad de Dinic es $\mathcal{O}(n^{2}m)$ debemos probar que compl(PB) = $\mathcal{O}(nm)$.

		Para hallar un flujo bloqueante:
		\begin{enumerate}
			\item Crear un NA: Como es con BFS es $\mathcal{O}(m)$
			\item Hallar bloqueante, denotemos:
				\begin{itemize}
					\item A: avanzar
					\item R: retorceder
					\item I: inicializar e incrementar
				\end{itemize}
		\end{enumerate}
		
		El paso bloqueante de Dinic luce de la forma:
		\begin{center}
			AA...AIAAARA...AIAARAAARR...IA...
		\end{center}
		
		subdiviadmoslo en palabras del tipo:
		\begin{center}
			\begin{itemize}
				\item AA...AI
				\item AA...AR
			\end{itemize}
		\end{center}
		
		donde las primeras son todas A pudiendo ser 0 la cantidad de la misma.
		
		Debemos determinar:
		\begin{enumerate}
			\item Cual es la complejidad de cada subpalabra.
				
				Recordemos que:
		\begin{itemize}
			\item A:  \{ $P[i+1] =$ algún elemento de $\Gamma^{+}(P[i])$ \\
						  $i = i+1$ \\
						$\Rightarrow$ A es $\mathcal{O}(1)$
						
			\item R: \{ borrar $P[i-1]$ del NA \\
						  $i = i-1$ \\
						$\Rightarrow$ R es $\mathcal{O}(1)$
						
			\item I: \{ Recorre dos veces un camino de longitud \textit{d} $=$ \textit{d(\textit{t})} \\
						$\Rightarrow$ R es $\mathcal{O}(\textit{d})$
		\end{itemize}
		
		Por lo tanto:
		\begin{equation}
			compl(A...AR) = \mathcal{O}(1) + ... \mathcal{O}(1) + \mathcal{O}(1)
									= \mathcal{O}(j)
		\end{equation}
		
		Pero como cada A hace $ i = i+1$ y tenemos $ 0 \leq \textit{i} \leq \textit{d} \Rightarrow \textit{j} \leq \textit{d}.$
		
		\begin{center}
			$ \; \therefore compl(A...AR) = \mathcal{O}(\textit{d})$
		\end{center}
		
		Similarmente: 		
		\begin{equation}
			compl(A...AI) = \mathcal{O}(1) + ... \mathcal{O}(1) + \mathcal{O}(1) \\
									= \mathcal{O}(d) + \mathcal{O}(d) \\
									= \mathcal{O}(d)
		\end{equation}
		
		Pero como cada A hace $ i = i+1$ y tenemos $ 0 \leq \textit{i} \leq \textit{d} \Rightarrow \textit{j} \leq \textit{d}.$
		
		\begin{center}
			$ \; \therefore compl(A...AR) = \mathcal{O}(\textit{d})$
		\end{center}			
			

			\item Cuantas palabras hay de cada tipo.
			
				
		\end{enumerate}
		
		
		
		
		
		
		
	\section{La complejidad de WAVE}

	
	\section{La distancia entre NA sucesivos aumenta}
	


\chapter{Parte B}

	\section{2-COLOR es polinomial}
	
	
	\section{Probar que:}
	
		\subsection{El valor de todo flujo es menor o igual que la capacidad de todo corte.}
		
		\subsection{Si el valor de un flujo es igual a la capacidad de un corte entonces el flujo es maximal y el corte minimal.}
		
		\subsection{Si un flujo es maximal entonces existe un corte con capacidad igual al valor del flujo.}
	
	
	\section{Complejidad del Hungaro es $\mathcal{O}(n^{4})$}
	
	
	\section{Teorema de Hall}
	
	
	\section{Teorema del matrimonio}
	
	
	\section{Si G es bipartito $\Rightarrow \chi '(G) = \Delta $}

	
	\section{Teorema cota de Hamming}
	
	
	\section{Sea H una matriz de chequeo de un código C, pruebe que:}
	
		\subsection{$\delta (C) =$ mínimo número de columnas linealmente dependientes de H}
		
		\subsection{Si H no tiene la columna cero ni columnas respetidas $\Rightarrow$ C corrige al menos un error}


	\section{Sea C un código cíclico de dimensión \textit{k} y longitud \textit{n} y sea $g(x)$ su polinomio generador, probar que:}
	
		\subsection{C está formado por los múltiplos de $g(x)$ de grado menor a \textit{n}}
		
		\subsection{El grado de $g(x)$ es $n-k$}
		
		\subsection{$g(x)$ divide a $1+x^{n}$}
		
		

\chapter{Parte C}

	\section{4-COLOR $\leq_{\textit{p}}$ SAT}
	
	\section{3-SAT es NP-Completo}
	
	\section{3-COLOR es NP-Completo}
	
	
\begin{thebibliography}{X}
\bibitem{Dan} \textsc{Curto Agustín },
<<Matemática Discreta II, apuntes de clase>>,
\textit{FaMAF, UNC}.
\bibitem{Baz} \textsc{Maximiliano Illbele},
<<Resumen de Discreta II, 16 de agosto de 2012>>,
\textit{FaMAF, UNC}.
\end{thebibliography}

\end{document}