\documentclass[12pt,a4paper]{report}
\usepackage[utf8]{inputenc}
\usepackage[spanish]{babel}
\usepackage{amsmath}
\usepackage{amsfonts}
\usepackage{amssymb}
\usepackage{lmodern}
\usepackage{amsmath}
\usepackage{enumerate}
\usepackage[left=2cm,right=2cm,top=2cm,bottom=2cm]{geometry}
\usepackage{graphicx}
\usepackage{algpseudocode}
\usepackage{stackrel}
\renewcommand{\theequation}{\arabic{equation}}
\newcounter{neq}
\providecommand{\abs}[1]{\lvert#1\rvert}
\newcommand{\QED}{\hfill \textit{\textbf{Q.E.D.}}}
\author{Agustin Curto, agucurto95@gmail.com}
\title{Resumen de teoremas para el final \\ de Matemática Discreta  II}
\date{2016}

\begin{document}
\maketitle
\tableofcontents


\chapter{Parte A}

	\section{La complejidad de EDMONS-KARP}
	
		\textbf{\underline{Teorema:}} La complejidad de $\langle E-K \rangle$ con $n = \abs{V}$ y $m = \abs{E}$ es $\mathcal{O}(nm^{2})$.
	
		\textbf{\underline{Prueba:}} Sean: $f_{0}, f_{1}, f_{2}, \; \dotsc$ \;  la sucesión de flujos creados por $\langle E-K \rangle$. Es decir, el paso \textit{k} crea $f_{k}$.
		
		\vspace{5mm}
		Para cada \textit{k} definimos funciones:
		
		\begin{itemize}
			\item $d_{k}(x) =$ \textquotedblleft distancia\textquotedblright $\;$ entre \textit{s} y x en el paso \textit{k} en caso de existir, si no $\infty$.
			\item $b_{k}(x) =$ \textquotedblleft distancia\textquotedblright $\;$ entre x y \textit{t} en el paso \textit{k} en caso de existir, si no $\infty$.
		\end{itemize}
	
		\textquotedblleft Distancia\textquotedblright: longitud del menor camino aumentante entre dos vértices.
		
		\vspace{5mm}
		Observaciónes:
			\begin{enumerate}
				\item
					\begin{itemize}
						\item $d_{k}(s) = 0$
						\item $b_{k}(t) = 0$
					\end{itemize}
				\item Sabemos que las distancias de $\langle E-K \rangle$ no disminuyen en pasos sucesivos, como esto será útil para esta demostración llamaremos $\circledast$ a la demostración de:
			\begin{center}
				$d_{k}(x) \leq d_{k+1}(x)$ \\
				$b_{k}(x) \leq b_{k+1}(x)$
			\end{center}
			\end{enumerate}

		Llamemos \textit{\underline{crítico}} a un lado disponible en el paso \textit{k} pero no disponible en el paso \textit{k+1}. Es decir, si \textit{xy} es un lado $\Rightarrow$ \textit{xy} se satura ó \textit{yx} se vacía en el paso \textit{k}.
		
		Supongamos que al construir $f_{k}$ el lado \textit{xy} se vuelve crítico, el camino: \textit{s} $\dotsc$ x,y $\dotsc$ \textit{t} se usa para construir $f_{k}$.

		\begin{eqnarray}
			 d_{k}(\textit{t}) & = & d_{k}(x)+ b_{k}(x) \\
			\nonumber & = & d_{k}(x)+b_{k}(y)+1
		\end{eqnarray}
		
		 Para que \textit{xy} pueda ser \textit{crítico} nuevamente debe ser usado en la otra dirección (\textit{i.e yx}). Sea \textit{j} el paso posterior a \textit{k} en el cual se usa el lado en la otra dirección, el camino \textit{s} $\dotsb$ y,x $\dotsb$ \textit{t} se usa para construir $f_{j}$.
		
		\begin{eqnarray}
			d_{j}(\textit{t}) & = & d_{j}(x)+ b_{j}(x) \\
			\nonumber & = & d_{j}(y)+1+b_{j}(x)
		\end{eqnarray}
		
		Entonces:
		
		\begin{equation*}
			\textup{De (1) y (2)} \Rightarrow
  			\left \lbrace
  			\begin{array}{l}
    		 d_{j}(x) = d_{j}(y)+1 \; \; \star \\
     		 d_{k}(y) = d_{k}(x)+1 \; \; \dag\\
  			\end{array}
  			\right.
		\end{equation*}
		
		Luego:

		\begin{eqnarray}
			\nonumber d_{j}(\textit{t}) & = & d_{j}(x)+ b_{j}(x) \\
			\nonumber & = & d_{j}(y)+1+b_{j}(x) \qquad\qquad\qquad\text{Por } \dag\\
			\nonumber & \geq & d_{k}(y)+1+b_{k}(x) \qquad\qquad\qquad\text{Por} \circledast \\
			\nonumber & = & d_{k}(x)+1+1+b_{k}(x) \;\qquad\qquad\text{Por} \star \\
			\nonumber & = & d_{k}(\textit{t})+2 \\
			\nonumber \Rightarrow d_{j}(\textit{t}) & \geq & d_{k}(\textit{t})+2
		\end{eqnarray}
		
		\vspace{5mm}
		Por lo tanto cuando un lado se vuelve crítico recien puede volver a saturarse cuando la distancia de \textit{s} a \textit{t} haya aumentado en por lo menos 2. Puede existir $\mathcal{O}(n/t)$ tales aumentos, es decir:
		
		\begin{center}
			\# Veces que un lado puede volverse crítico $= \mathcal{O}(n)$. 
		\end{center}				
		
		\begin{eqnarray}
			 \nonumber \; \therefore Complejidad(\langle E-K \rangle) &=& (\# pasos) * Compl(1 \; \textit{paso}) \\
			 \nonumber &=& (\# \textup{veces que un lado se vuelve crítico}) * (\# lados) * Compl(BFS) \\
			\nonumber  &=& \mathcal{O}(n) * \mathcal{O}(m)* \mathcal{O}(m) \\
			\nonumber &=& \mathcal{O}(nm^{2})
		\end{eqnarray}
			
		
	\section{Las distancias de Edmonds-Karp no disminuyen en pasos sucesivos}
		\underline{Teorema:} Sean: $f_{0}, f_{1}, f_{2}, \; \dotsc$ \;  la sucesión de flujos creados por $\langle E-K \rangle$. Es decir, el paso \textit{k} crea $f_{k}$.
		
		\vspace{5mm}
		Para cada \textit{k} definimos funciones:
		
		\begin{itemize}
			\item $d_{k}(x) =$ \textquotedblleft distancia\textquotedblright $\;$ entre \textit{s} y x en el paso \textit{k} en caso de existir, si no $\infty$.
			\item $b_{k}(x) =$ \textquotedblleft distancia\textquotedblright $\;$ entre x y \textit{t} en el paso \textit{k} en caso de existir, si no $\infty$.
		\end{itemize}
	
		\textquotedblleft Distancia\textquotedblright: longitud del menor camino aumentante entre dos vértices.
		
		\vspace{5mm}
		Queremos probar que:
		\begin{enumerate}
			\item $d_{k}(x) \leq d_{k+1}(x)$
			\item $b_{k}(x) \leq b_{k+1}(x)$
		\end{enumerate}
		
		\underline{Prueba:} Lo probaremos por inducción y solo para $d_{k}$ ya que para $b_{k}$ la prueba es análoga.
		
		\begin{center}
			HI: $H(i) = \lbrace\forall_{z}: d_{k+1}(z) \leq \textit{i}, \, vale \; d_{k}(z) \leq d_{k+1}(z)  \rbrace$
		\end{center}
		
		\begin{enumerate}
			\item Caso Base: \begin{tabular}{|c|} \hline i = 0 \\ \hline \end{tabular} \qquad	$H(0) = \lbrace\forall_{z}: d_{k+1}(z) \leq 0, \, d_{k}(z) \leq d_{k+1}(z)  \rbrace$
			
			Pero $d_{k+1}(z) \leq 0 \Rightarrow z = \textit{s}$
			
			\begin{eqnarray}
				\nonumber d_{k}(z) & = & d_{k}(\textit{s}) \\
				\nonumber & = & 0 \\
				\nonumber & \leq & d_{k+1}(\textit{s}) \\
				\nonumber & \leq & d_{k+1}(z) \\
				\nonumber \Rightarrow d_{k}(z) & \leq & d_{k+1}(z)
			\end{eqnarray}
			
			\item Caso Inductivo: Supongamos ahora que vale H(\textit{i}), veamos que vale H(\textit{)i+1}.
			
			Sea \textit{z} con $d_{k+1}(z) \leq \textit{i+1}$, si $d_{k+1}(z) \leq \textit{i}$ vale H(\textit{i}) para \textit{z}. 
			
			\begin{center} $\therefore \; d_{k+1}(z) \leq d_{k+1}(z) $ \end{center}
			
			Supongamos que \begin{tabular}{|c|} \hline $d_{k+1}(z) = i+1$ \\ \hline \end{tabular}
			
			Entonces existe un camino aumentante, relativo a $f_{k}$, de la forma: $s = z_{0}, \; z_{1}, \; \dotsc \; z_{i}, \; z_{i+1} = z$.
			
			Sea \begin{tabular}{|c|} \hline $x = z_{i}$ \\ \hline \end{tabular}
			
			\begin{itemize}
				\item \underline{Caso 1:} Existe algun camino aumentante, relativo a $f_{k-1}$ de la forma $s, \; \dotsc \; x, \; z$.
				$\Rightarrow \begin{tabular}{|c|} \hline $d_{k}(x) \leq d_{k}(x) + 1$ \\ \hline \end{tabular}$
				
				\vspace{2mm}
				Pues al haber un camino $\underbrace{s, \; \dotsc \; x,}_{d_{k}(x)} \; z$, llamemosle A, de longitud $d_{k}(x) + 1$ entre \textit{s} y \textit{z}, sabemos que el minimo de todos los caminos de \textit{s} a \textit{z} seran $\leq$ A.
				
				\item Caso 2: No existe un camino aumentante, relativo a $f_{k-1}$, pero si existe un camino aumentante relativo a $f_{k}$. Por lo tanto el lado \textit{xz} no esta \textquotedblleft disponible\textquotedblright \; en el paso \textit{k}, ya que \textit{xz} está saturado \textit{zx} está vacío relativo a $f_{k-1}$. Para construir $f_{k}$ usamos un camino de la forma $s, \; \dotsc \; z, \; x$. Es decir:
				
				\begin{enumerate}[1)]
					\item $f_{k-1}(xz) = C(xz)$ pero $f_{k}(xz) < C(xz), \; f_{k}$ devuelve flujo por \textit{xz} ó
					\item $f_{k-1}(zx) = 0$ pero $f_{k}(zx) > 0, \; f_{k}$ manda flujo por \textit{zx}.
				\end{enumerate}
				
				Como $\langle E-K \rangle$ funciona con BFS, ese camino usado pra construir $f_{k}$ debe ser de longitud mínima. Es decir: \; $d_{k}(x) = d_{k}(z) + 1$
				
				\begin{eqnarray}
					\nonumber d_{k}(z) & = & d_{k}(x) - 1 \\
					\nonumber & \leq & d_{k}(x) + 1
				\end{eqnarray}				
			\end{itemize}
			
			\underline{Conclusión:} En cualquiera de los dos casos tenemos:
			\begin{center}
			\begin{tabular}{|c|} \hline $d_{k}(x) \leq d_{k}(x) + 1$ \\ \hline \end{tabular}
			\end{center}
			
			Ahora bien: \qquad $d_{k+1}(x) = d_{k+1}(z_{i}) = i \; \; \Rightarrow \; \;$ vale H(\textit{i}) para x. \\
			$ \; \qquad \qquad \therefore d_{k}(z) \leq d_{k+1}(x)$
			
			\begin{eqnarray}
					\nonumber d_{k+1}(x) & = & d_{k+1}(z_{i}) \\
					\nonumber & = & i \\
					\nonumber & \Rightarrow & \textup{H(\textit{i}) vale para x.} \\
					\nonumber \therefore \; d_{k}(z) & \leq & d_{k+1}(x)
				\end{eqnarray}
			
			Por lo tanto:
			\begin{eqnarray}
					\nonumber d_{k}(z) & \leq & d_{k}(x) + 1 \\
					\nonumber & \leq & d_{k+1}(x) + 1 \\
					\nonumber & = & i+1 \\
					\nonumber & = & d_{k+1}(z) \\
					\nonumber & \Rightarrow & \textup{H(\textit{i+1}) vale.}
				\end{eqnarray}
		\end{enumerate}


	\section{La complejidad de DINIC}

		\underline{Teorema:} La complejidad del algoritmo de Dinic es $\mathcal{O}(n^{2}m)$.
		
		\underline{Prueba:} Como Dinic es un algoritmo que trabaja con networks auxiliares y vimos que la distancia entre \textit{s} y \textit{t} en networks auxiliares consecutivos aumenta y puede ir a lo sumo entre 1 y $n-1$ entonces hay a lo sumo $\mathcal{O}(n)$ networks auxiliares.

		\begin{center}
			Complejidad(Dinic)$ = \mathcal{O}(n) *$ Compl(Hallar un flujo bloqueante en un NA con Dinic)
		\end{center}
		
		Para probar que la complejidad de Dinic es $\mathcal{O}(n^{2}m)$ debemos probar que complejidad del paso bloqueante es $\mathcal{O}(nm)$.
		
		Sea:
		\begin{itemize}
			\item A $=$ Avanzar()
			\item R $=$ Retroceder()
			\item I $=$ Incrementar\_Flujo + Inicialización ($\mathcal{O}(1)$) 
		\end{itemize}

		Una corrida de Dinic luce como:
		\begin{center}
			AA$\; \dotsc \;$AIAAARA$\; \dotsc \;$AIAARAAARR$\; \dotsc \;$IA$\; \dotsc \;$
		\end{center}
		
		Dividamos la corrida en subpalabras del tipo:
		\begin{center}
			\begin{itemize}
				\item[$*$] $\underbrace{AA \; \dotsc \; A}_{Todas A's}I$
				\item[$*$] $\underbrace{AA \; \dotsc \; A}_{Todas A's}R$
			\end{itemize}
		\end{center}
		
		Nota: el número de A's puede ser 0.
		
		Debemos determinar:
		\begin{enumerate}
			\item Cual es la complejidad de cada subpalabra.
			\item Cuantas palabras hay de cada tipo.
		\end{enumerate}
		
		\textbf{Complejidad de cada subpalabra}
		
		Recordemos que:
		\begin{equation*}
			\textup{A:}
  			\left[
  			\begin{array}{l}
    		 P[i+1] = \textup{algún elemento de } \Gamma^{+}(P[i]) \\
     		 i = i+1\\
  			\end{array}
  			\right.
		\end{equation*}
		\begin{center}
			$\Rightarrow$ A es $\mathcal{O}(1)$
		\end{center}
		
		\begin{equation*}
			\textup{R:}
  			\left[
  			\begin{array}{l}
    		 P[i+1] = \textup{borrar } P[i-1]P[i] \textup{ del NA} \\
     		 i = i-1\\
  			\end{array}
  			\right.
		\end{equation*}
		\begin{center}
			$\Rightarrow$ R es $\mathcal{O}(1)$
		\end{center}
		
		\begin{equation*}
			\textup{I:}
  			\left[
  			\begin{array}{l}
    		 P[i+1] = \textup{recorre 2 veces, un camino de longitud } d = d(t)
  			\end{array}
  			\right.
		\end{equation*}
		\begin{center}
			$\Rightarrow$ I es $\mathcal{O}(d)$
		\end{center}
		
		Por lo tanto:
		\begin{eqnarray}
			\nonumber Compl(\underbrace{A \; \dotsc \; A}_{j \; veces}R) & = & \underbrace{\mathcal{O}(1) + \; \dotsc \; \mathcal{O}(1)}_{j \; veces} + \mathcal{O}(1) \\
			\nonumber & = & \mathcal{O}(j) + \mathcal{O}(1) \\
			\nonumber & = & \mathcal{O}(j)
		\end{eqnarray}
		
		Pero como cada A hace $ i = i+1$ y tenemos $ 0 \leq \textit{i} \leq \textit{d} \; \Rightarrow \; \textit{j} \leq \textit{d}.$
		
		\begin{center}
			$\therefore \; Compl(A \; \dotsc \; AR) = \mathcal{O}(\textit{d})$
		\end{center}
		
		Similarmente: 		
		\begin{eqnarray}
			\nonumber Compl(A \; \dotsc \; AI) & = & \underbrace{\mathcal{O}(1) + \; \dotsc \; \mathcal{O}(1)}_{\leq \; d \; veces} + \mathcal{O}(1) \\
			\nonumber & = & \mathcal{O}(d) + \mathcal{O}(1) \\
			\nonumber & = & \mathcal{O}(d)
		\end{eqnarray}
				
		\textbf{Cantidad de subpalabras}
		\begin{itemize}
			\item R tiene la instrucción "\textbf{borrar lado}". Como los lados borrados quedan borrados hay a lo sumo \textit{m} R's, es decir:
			\begin{center}
				$ \therefore \; \# (A \; \dotsc \; AR's) \leq m $
			\end{center}
			
			\item I tiene también línes de la forma:
				\begin{algorithmic}
					\If{$\dotsc$}
    					\State{borrar lado}
    				\EndIf
				\end{algorithmic}
			 	
			 	Lo que está dentro del \textbf{if} se cumple al menos una vez, es decir:
			 	\begin{center}
					$ \therefore \; \# (A \; \dotsc \; AI's) \leq m $
				\end{center}
			
			Este análisis muestra que:
			\begin{center}
					$ \therefore \; \# (A \; \dotsc \; AR's) + \# (A \; \dotsc \; AI's) \leq m $
				\end{center}
			
			Por lo tanto hay $\leq m$ palabras, cada una de complejidad $\mathcal{O}(d)$.

			\begin{eqnarray}
			\nonumber \therefore \;  Compl(Paso \; Bloqueante) & = & \mathcal{O}(m) + \mathcal{O}(md) \\
			\nonumber & = & \mathcal{O}(mn)
		\end{eqnarray}
		
		ya que $d \leq n$.			
		\end{itemize}
		

	\section{La complejidad de WAVE}

		\underline{Teorema:} La complejidad del algoritmo de Wave es $\mathcal{O}(n^{3})$.
		
		\underline{Prueba:} Como Wave es un algoritmo que trabaja con networks auxiliares y vimos que la distancia entre \textit{s} y \textit{t} en networks auxiliares consecutivos aumenta y puede ir a lo sumo entre 1 y $n-1$ entonces hay a lo sumo $\mathcal{O}(n)$ networks auxiliares.

		\begin{center}
			Complejidad(Wave)$ = \mathcal{O}(n) *$ Compl(Hallar un flujo bloqueante en un NA con Wave)
		\end{center}
		
		Para probar que la complejidad de Wave es $\mathcal{O}(n^{3})$ debemos probar que complejidad del paso bloqueante es $\mathcal{O}(n^{2})$. El paso bloqueante de Wave consiste en una serie de:
		
		\begin{itemize}
			\item Olas hacia adelante: Sucesión de \textbf{forwrdbalance} (FB)
			\item Olas hacia atrás: Sucesión de \textbf{backwardbalance} (BB)
		\end{itemize}
		
		Cada FB y BB es una sucesión de \textquotedblleft \textbf{buscar vecinos}\textquotedblright y \textquotedblleft \textbf{procesar}\textquotedblright el lado resultante. Estos "procesamientos" \; son complicados pero $ \mathcal{O}(1)$.
		
		\begin{center}
		$ \therefore \; Compl(Paso \; Bloqueante) = \# "procesamientos"$ de lados
		\end{center}
	
		Los "procesamientos" \; de lados los podemos dividir en dos categorías:
		\begin{enumerate}
			\item Aquellos procesamientos que saturan o vacian el lado. Denotaremos \textquotedblleft T\textquotedblright \; al número de estos procesamientos.
			\item Aquellos procesamientos que no saturan ni vacian el lado. Denotaremos \textquotedblleft Q\textquotedblright \; al número de estos procesamientos.
		\end{enumerate}
	
		Por lo tantos queremos acotar $T + Q$.
		
		\vspace{5mm}
		\textbf{Complejidad de T:}
		\begin{itemize}
			\item ¿Puede un lado \textit{xy} saturado volver a saturarse?
			
			Para poder volver a \underline{saturarse} primero tiene que vaciarse auque sea un poco, es decir, antes de poder volver a saturarlo \textquotedblleft \textit{y}\textquotedblright \; debe devolver flujo a \textquotedblleft \textit{x}\textquotedblright \;, pero para que en Wave \textquotedblleft \textit{y}\textquotedblright \; le devuelva flujo a \textquotedblleft \textit{x}\textquotedblright \;  debe ocurrir que \textquotedblleft \textit{y}\textquotedblright \; este bloqueado (porque BB(y) solo se ejecuta si \textquotedblleft \textit{y}\textquotedblright \; está bloqueado), pero si \textquotedblleft \textit{y}\textquotedblright \; está bloqueado \textquotedblleft \textit{x}\textquotedblright \; no puede mandarle flujo nunca más.
			
			\begin{center}
				$ \therefore \;$ \textit{xy} no puede resaturarse
			\end{center}
			
			\underline{Conclusión 1:} Los lados se saturan solo una vez.
			
			\item ¿Puede un lado \textit{xy} vaciado completamente volver a vaciarse?
			
			Para poder volver a \underline{vaciarse} como está vacío completamente, primero hay que mandar flujo, pero si lo vacié \textquotedblleft \textit{y}\textquotedblright \; está bloqueado por lo que \textquotedblleft \textit{x}\textquotedblright \; no puede mandar flujo.
			
			\begin{center}
				$ \therefore \;$ \textit{xy} no puede volver a vaciarse
			\end{center}
			
			\underline{Conclusión 2:} Los lados se vacían completamente a lo sumo una vez.
			
			Las conclusiones (1) y (2) implican que \begin{tabular}{|c|} \hline $T \leq 2 \; m$ \\ \hline \end{tabular}
		\end{itemize}
	
		\textbf{Complejidad de Q:}
	
			En cada FB a lo sumo un lado no se satura y en cada BB a lo sumo un lado no se vacía completamente.
			
			\vspace{2mm}
			$\therefore \;$ Q $\leq \#$ Total de FB's y BB's 
	
			\begin{itemize}
				\item $\#$ FB's en cada ola hacia adelante es $\leq$ \textit{n} (un FB por vértice)
				\item $\#$ BB's en cada ola hacia atrás es $\leq$ \textit{n}

				$\therefore$ Total de FB's y BB's $\leq 2 \; n \; \#$Total de ciclos de \textquotedblleft ola adelante $-$ ola hacia atrás\textquotedblright
 			\end{itemize}
	
			Ahora bien, en cada ola hacia adelante, pueden o no, bloquearse algunos vértices. Si no se bloquea ningún vértice entonces todos los vértices ($\neq$ \textit{s, t}) quedan balaceados por lo que estamos en la última ola. Luego en toda ola que no sea la última se bloquea al menos un vértice ($\neq$ \textit{s, t}).
			
			\begin{center}
			$\therefore \; \#$Total de ciclos es $\leq (n-2)+1 = n-1$ \\
			$\Rightarrow$ \begin{tabular}{|c|} \hline $Q \; \leq 2 n (n-1) = \mathcal{O}(n^{2})$ \\ \hline \end{tabular}
			\end{center}
	
			\begin{eqnarray}
				\nonumber \therefore \; T + Q & \leq & 2 m + \mathcal{O}(n^{2}) \\
				\nonumber & = & \mathcal{O}(m) + \mathcal{O}(n^{2}) \\
				\nonumber & = & \mathcal{O}(n^{2})
			\end{eqnarray}


	\section{La distancia entre NA sucesivos aumenta}
	
		\underline{Teorema:} Sea A un NA (network auxiliar) y sea $A^{*}$ el siguiente NA. Sean d(x) y $d^{*}(x)$ las distancias de \textit{s} a \textit{t} en A y $A^{*}$ respectivamente, entonces: $d(t) < d^{*}(t)$.
		
		\underline{Prueba:} Como A y $A^{*}$ se construyen con BFS sabemos que $d(t) \leq d^{*}(t)$ pero queremos ver el $<$.
		
		Sea:
		\begin{center}
			$s = x_{0}, \; x_{1}, \; \dotsc \; t = x_{r}$
		\end{center}
	
		un camino dirigido en $A^{*}$.
	
		Ese camino \begin{tabular}{|c|} \hline No existe \\ \hline \end{tabular} en A ya que para pasar de A a $A^{*}$ debemos bloquear todos los caminos dirigidos de A. Por lo tanto si ese camino estuviese en A, Dinic lo habría bloqueado y no estaría en $A^{*}$.
		
		\textbf{¿Cuáles son las razones posibles para que no esté en A?}
		\begin{enumerate}
			\item Puede faltar un vértice, es decir $\exists \; i \; : \; x_{i} \; \nexists \; V(A)$ entonces:
			\begin{eqnarray}
				\nonumber d(t) & \leq & d(x_{i}) \\
				\nonumber & \leq & d^{*}(x_{i}) \\
				\nonumber & = & i < r \\
				\nonumber & = & d^{*}(t) \\
				\nonumber \begin{tabular}{|c|} \hline $\therefore \; d(t) < d^{*}(t)$ \\ \hline \end{tabular}
			\end{eqnarray}
			
			\item Están todos los vértices pero falta una arista, es decir $\exists \; i \; : \; x_{i}x_{i+1} \; \nexists \; E(A)$.
			\begin{enumerate}[a)]
				\item $x_{i}x_{i+1}$ no está porque corresponde a un lado vacío o saturado en NA, es decir $x_{i}x_{i+1}$ no está en el recidual que dá origen a A pero si está en el residual que dá origen a $A^{*}$.
				
				Para que esto pase se tiene que haber usado el lado $x_{i+1}x_{i}$ en A. Luego podemos cocluir, por la prueba de $\langle E-K \rangle$ que:
				\begin{center}
					$\qquad \; d^{*}(t) \geq d(t) + 2 > d(t) $
					
					\vspace{2mm}
					\begin{tabular}{|c|} \hline $\therefore \; d(t) < d^{*}(t)$ \\ \hline \end{tabular}
				\end{center}
				
				\item $x_{i}x_{i+1}$ si está en el residual pero:
				\begin{tabular}{|c|} \hline $d(x_{i+1}) \neq d(x_{i}) +1 $ \\ \hline \end{tabular} (1)
				
				\vspace{5mm}
				Pero como $x_{i}x_{i+1}$ está en el residual entonces:
				\begin{tabular}{|c|} \hline $d(x_{i+1}) \leq d(x_{i}) +1 $ \\ \hline \end{tabular} (2)
				
				\vspace{5mm}
				De (1) y (2) tenemos que: \begin{tabular}{|c|} \hline $d(x_{i+1}) < d(x_{i}) +1 $ \\ \hline \end{tabular} $\circledast$
			
				Entonces:
				\begin{eqnarray}
				\nonumber d(t) & = & d(x_{i+1}) + b(x_{i+1}) \qquad\;\;\;\;\;\textup{Por }\langle E-K \rangle\\
				\nonumber & \leq & d(x_{i+1}) + b^{*}(x_{i+1}) \qquad\;\;\;\;\textup{Por }\langle E-K \rangle\\
				\nonumber & < & d(x_{i}) +1 + b^{*}(x_{i+1}) \qquad\;\textup{Por } \circledast \\
				\nonumber & \leq & d^{*}(x_{i}) + 1 + b^{*}(x_{i+1}) \qquad\textup{Por }\langle E-K \rangle \\
				\nonumber & = & d^{*}(x_{i+1}) + b^{*}(x_{i+1}) \qquad\;\;\;\textup{Por } (\dag) \\
				\nonumber & = & d^{*}(t)
				\end{eqnarray}
				$\qquad\qquad\qquad\;\;\;$
				\begin{tabular}{|c|} \hline $\therefore \; d(t) < d^{*}(t)$ \\ \hline \end{tabular}

				($\dag$): Ya que $s, \; x_{1}, \; \dotsc \; x_{r}$
			\end{enumerate}
		\end{enumerate}
	
	

\chapter{Parte B}

	\section{2-COLOR es polinomial}
	
	
	\section{Teorema Max-Flow Min-Cut}
		\underline{Teorema:}
		\begin{enumerate}[a)]
			\item Si \textit{f} es flujo y S es corte $\Rightarrow$ V(\textit{f}) $\leq$ Cap(S).
			\item Si V(\textit{f}) $=$ Cap(S) $\Rightarrow$ \textit{f} es maximal y S es minimal.
			\item Si \textit{f} es maximal $\Rightarrow \exists$ S con V(\textit{f}) $=$ Cap(S).
		\end{enumerate}
		
		\underline{Prueba:} Demostraremos primero que $V(\textit{f}) = f(S, \overline{S}) - f(\overline{S},S)$ donde \textit{f} es un flujo y S un corte. Esto nos ayudará en la demostración del ítem a).
		
		\vspace{5mm}
		\underline{Observemos que:}
		\begin{itemize}
			\item $f(A \cup B, C) = f(A,C) + f(B,C):$ A y B disjuntos.
			\item $f(A, B \cup C) = f(A,B) + f(A,C):$ B y C disjuntos.
			\item $f(A, B) = \sum_{\begin{subarray}{l} x \in A\\
y \in B\end{subarray}} f(x, y)$.
		\end{itemize}

		Sea x $\in$ S $\Rightarrow$ x $\neq$ \textit{t}.
		\begin{equation*}
			\textup{f(x, V) - f(V, x)} =
  			\left\lbrace
  			\begin{array}{l}
    		 V(f) \; Si \; x= \textit{s} \\
     		 0 \; \; \; \; \; \; \;  Si \; x \neq \textit{s} \; pues \; \textit{t} \notin S \\
  			\end{array}
 			 \right.
		\end{equation*}
		
		Luego:
		\begin{eqnarray}
			\nonumber \sum_{x \in S}(f(x, V) - f(V, x)) & = & 0 + 0 \dotsb + V(f) \\
			\nonumber & = & V(f)
		\end{eqnarray}
		
		\begin{flushright}
			\begin{eqnarray}
			\nonumber V(f) & = & \sum_{x \in S}f(x, V) - \sum_{x \in S}f(V, x) \\
			\nonumber & = & f(S, V) - f(V, S) \;\;\qquad\qquad\qquad\qquad\qquad\qquad\text{Por observación} \\
			\nonumber & = & f(S, S \cup \overline{S}) - f(S \cup \overline{S}, S) \;\;\;\;\qquad\qquad\qquad\qquad\text{Ya que } V = S \cup \overline{S} \\
           \nonumber & = & f(S, S) + f(S, \overline{S}) - f(S, S) -  f(\overline{S}, S) \qquad\qquad\text{Por observación} \\
           \nonumber & = & \begin{tabular}{|c|} \hline $f(S, \overline{S}) - f(\overline{S}, S)$ \\ \hline \end{tabular} \; (\star)
		\end{eqnarray}
		\end{flushright}
		
		\textbf{a) \textit{f} es flujo y S es corte $\Rightarrow$ V(\textit{f}) $\leq$ Cap(S).}
		
			\begin{center}
				$V(f) \stackbin{Por \; (\star)}{=} f(S, \overline{S})\underbrace{-\underbrace{f(\overline{S}, S)}_{\geq 0}}_{\leq 0}$ \\
				
				\vspace{5mm}
				$\Rightarrow V(f ) \leq f(S, \overline{S}) \leq C(S, \overline{S}) =$ Cap(S)
			\end{center}
			
		\textbf{b) V(\textit{f}) $=$ Cap(S) $\Rightarrow$ \textit{f} es maximal y S es minimal.}
			
			\vspace{2mm}
			Supongamos que V(\textit{f}) $=$ Cap(S). Sea \textit{g} un flujo cualquiera y T un corte cualquiera.
			\begin{itemize}
				\item $V(g) \stackbin{Por \; a)}{\leq} Cap(S) = V(f) \Rightarrow$ f es maximal 
				\item $Cap(T) \stackbin{Por \; a)}{\leq} V(f) = Cap(S) \Rightarrow$ S es minimal
			\end{itemize}
		
		\vspace{5mm}
		\textbf{c) \textit{f} es maximal $\Rightarrow \exists$ S con V(\textit{f}) $=$ Cap(S).}
			
			\vspace{2mm}
			Sea $S = \{s\} \cup \{x \; : \; \exists \;$ camino aumentante realtivo a \textit{f} entre \textit{s} y \textit{t}$\}$
			
			\vspace{5mm}
			¿$t \in S$?

			\underline{Si \textit{t} estuviera en S:} existiría un camino aumentante entre \textit{s} y \textit{t}.
				
				Por el teorema del camino aumentante podemos construir un  flujo \textit{g} tal que:
				\begin{center}
					$V(g) = V(f) + \epsilon \textup{ para algun } \epsilon > 0$
					
					$\therefore V(g) > V(f)$ \textbf{Absurdo} pues \textit{f} es maximal.
				\end{center}
				
				$\therefore t \notin S \Rightarrow S$ es corte.
				
				\begin{center}
					Solo resta ver que: $V(f) = Cap(S)$
				\end{center}
				
				Por $(\star): \; V(f) =  \underbrace{f(S, \overline{S})}_{(1)} - \underbrace{f(\overline{S}, S)}_{(2)}$
				
				\vspace{2mm}
				Analicemos (1) y (2)
			\begin{enumerate}[(1)]
				\item $f(S, \overline{S}) = \sum_{\begin{subarray}{l} \; x \in S\\
\; y \in \overline{S} \\ xy \in E\end{subarray}} f(\overrightarrow{xy}) $

				$x \in S \Rightarrow \exists$ camino aumentante $ x_{0} = s, \; x_{1}, \dotsc x_{r} = x$.

				$y \in \overline{S} \Rightarrow \nexists$ camino aumentante entre \textit{s} y \textit{y}. En particular $x_{0} = s, x_{1}= x, \dotsc y$ no es camino aumentante.
				
				\vspace{3mm}
				$\Rightarrow f(\overrightarrow{xy}) = Cap(\overrightarrow{xy}) \qquad \forall x \in S, \forall y \in \overline{S} : \overrightarrow{xy} \in E.$ 
				
				\vspace{3mm}
				$\Rightarrow f(S, \overline{S}) = \sum_{\begin{subarray}{l} \; x \in S\\  \; y \in \overline{S} \\ xy \in E\end{subarray}} f(\overrightarrow{xy}) = \sum_{\begin{subarray}{l} \; x \in S\\  \; y \in \overline{S} \\ xy \in E\end{subarray}} Cap(\overrightarrow{xy}) = Cap(S, \overline{S}) = Cap(S)$				

				\item $f(\overline{S}, S) = \sum_{\begin{subarray}{l} \; x \in \overline{S}\\
\; y \in S \\ xy \in E\end{subarray}} f(\overrightarrow{xy}) $

				$x \in \overline{S} \Rightarrow \nexists$ camino aumentante entre \textit{s} y \textit{x}.

				$y \in S \Rightarrow \exists$ camino aumentante $ y_{0} = s, \; y_{1}, \dotsc y_{r} = y$.
				
				En particular $s \dotsc y, y \dotsc x$ no es camino aumentante. $\Rightarrow f(\overrightarrow{xy}) = 0 \qquad \forall x \in S, \forall y \in \overline{S} : \overrightarrow{xy} \in E$.
				
				\vspace{3mm}
				$\Rightarrow f(\overline{S}, S) = 0$
			\end{enumerate}
			
			Luego de (1) y (2)
			
			\begin{eqnarray}
				\nonumber V(f) &=& f(S, \overline{S}) - f(\overline{S}, S)\\
				\nonumber &=& Cap(S) - 0 \\
				\nonumber &=& Cap(S)				
			\end{eqnarray}

	\section{Complejidad del Hungaro es $\mathcal{O}(n^{4})$}
	
		\underline{Teorema:} La complejidad del algoritmo Húngaro es $\mathcal{O}(n^{4})$.
		
		\underline{Prueba:} 
		
			\begin{enumerate}
				\item La complejidad del matching inicial es $\mathcal{O}(n^{2})$, ya que:
				
					Restar mínimo de cada fila:
					\begin{center}
						$(\underbrace{\mathcal{O}(n)}_{calcular \; min} + \underbrace{\mathcal{O}(n)}_{restar \; min}) * \underbrace{n}_{\# filas} = \mathcal{O}(n^{2})$
					\end{center}
				
				Idem para las columnas.
				
				\item Llamemos \textbf{extender} el matching, a incrementar su número de filas en 1, i.e agregar una fila más al matching.
				\begin{center}
					$ \# \; extensiones \; de \; matching = \mathcal{O}(n)$
				\end{center}
				
				Resta ver la complejidad de cada \textbf{extender}.
				
				\item En cada extensión vamos a ir revisando filas y columnas, donde escanear una fila es $\mathcal{O}(n)$ y se realizan \textit{n} escaneos, por lo que sería $\mathcal{O}(n^{2})$ sin considerar que se debe realizar un cambio de matriz.
				
				Hacer un \textbf{cambio de matriz} es $\mathcal{O}(n^{2})$, ya que:				
				\begin{itemize}
					\item Buscar $\textit{m} = \min S \; x \; \overline{\Gamma(S)} \rightarrow \mathcal{O}(n^{2})$
					\item Restar \textit{m} de $S \rightarrow \mathcal{O}(n^{2})$
					\item Sumar \textit{m} a $\Gamma(S) \rightarrow \mathcal{O}(n^{2})$
				\end{itemize}
				
				Luego la implementación NAIVE lanzaría nuevamente el algoritmo desde cero. La forma correcta es continuar con el matching que teniamos, ya que el mismo no se pierde.
				
				Si lo hacems así, ¿Cuántos \textbf{cambios de matriz} habrá antes de extender un matching nuevamente?
				
				\underline{\textbf{Lema Interno:}} Luego de un \textbf{cambio de matriz}, se extiende el matching o bien se aumenta S.
				
				\underline{\textbf{Prueba:}} 			
				
				\begin{center}$
				\; \; \left(
				\begin{array}{r | r}
					\text{{\huge A}} & \text{{\huge A}} \\
					\hline
					\text{{\huge B}} & \text{{\huge C}}
				\end{array}
				\right)
				{\begin{subarray}{l} \; \overline{\text{{\large S}}} \\ \\ \; {\text{{\large S}}} \end{subarray}} \linebreak
				\overline{\Gamma(S)} \; \Gamma(S)
				$
				\end{center}
				
				\underline{Referencias:}
				\begin{itemize}
					\item A: puede haber ceros.
					\item B: no hay ceros, no hay matching.
					\item C: ceros del matching.
				\end{itemize}
				
				Al restar $\textit{m} = \min S \; \Gamma(S)$ de las filas de S, habrá un nuevo cero en alguna fila \textit{i} ($\in S$) y columna \textit{j} ($\in \Gamma(S)$) entonces la columna se etiquetará con \textit{i} y se revisará.
				Tenemos dos resultados posible:
				\begin{enumerate}
					\item \textit{j} está libre (i.e no forma parte del matching) $\Rightarrow$ extendemos el matching.
					\item \textit{j} forma parte de matching $\Rightarrow \exists$ fila \textit{k} matcheada con \textit{j}. En este caso, la fila \textit{k} se etiquetará con \textit{j}, por lo que el "nuevo" $\; S \geq S \cup \{\textit{k}\}$.
					
					\vspace{5mm}
					Entonces se termina con una extensión o se produce un nuevo S de cardinalidad, al menos $\lvert S \rvert + 1$.
				\end{enumerate}
				
				\textbf{Fin Lema Interno}
				
				Luego como $\lvert S \rvert$ solo puede crecer $\mathcal{O}(n)$ veces, tenemos que hay a lo sumo \textit{n} \textbf{cambios de matriz} antes de extender el matching. Entonces:
				
							\begin{center}
								Complejidad(1 Extensión) $= \underbrace{\mathcal{O}(n)}_{\# CM} * \underbrace{\mathcal{O}(n^{2})}_{Compl(CM)} + \underbrace{\mathcal{O}(n^{2})}_{Busqueda \; \textit{n} \; filas \; x \; \textit{n} \; columnas }$ \\
								\vspace{5mm}
								$\therefore$ Complejidad(Húngaro) $= \underbrace{\mathcal{O}(n^{2})}_{Matching \; inicial} + \underbrace{\mathcal{O}(n)}_{\#extensiones} * \underbrace{\mathcal{O}(n^{3})}_{Compl(extension)}) = \; \mathcal{O}(n^{4})$
							\end{center}				
			\end{enumerate}
	
	\section{Teorema de Hall}
	
		\underline{Teorema:} Sea G = (X $\cup$ Y, E) grafo bipartito $\Rightarrow \exists$ matching completo de X a Y $\Leftrightarrow \; \lvert S \rvert \leq \lvert \Gamma(S) \rvert \; \forall S \subseteq X$.
		
		\vspace{3mm}
		\underline{Prueba:}

			$\Rightarrow)$ Si M es matching completo de X a Y entonces oberservemos que M induce una función inyectiva de X a Y.
			
			\begin{center}
				$f(x) = $ único y : xy $\in$ M.
			\end{center}
			
			\begin{enumerate}
				\item Si $S \subseteq X \Rightarrow \lvert S \rvert = \lvert \Gamma(S) \rvert$. 

				Además por definición de f, $f(x) \in \Gamma(x)$.
				\item Si $x \in S \Rightarrow f(x) \in \Gamma(S)
				\Rightarrow f(S) \subseteq \Gamma(S)$.
			\end{enumerate}
			
			De \textcircled{1} y \textcircled{2} $\Rightarrow \lvert S \rvert \leq \lvert \Gamma(S) \rvert$.
			
			\vspace{5mm}
			$\Leftarrow)$ Supongamos que no es cierto, entonces G es bipartito con $\lvert S \rvert \leq \lvert \Gamma(S) \rvert \; \forall S \subseteq X$ pero no tiene matching completo de X a Y. Esto equivalente a ver que: Si $\nexists$ un matching completo $\Rightarrow \exists \; S \subseteq X : \lvert S \rvert > \lvert \Gamma(S) \rvert$.
			
			\vspace{5 mm}
			Corramos el algoritmo para hallar matching. Al finalizar habrá filas sin matcher (las de \textit{s}).
			
			\vspace{5mm}
			Sean:
			\begin{itemize}
				\item $S_{0} =$ filas sin matchear.
				\item $T_{1} = \Gamma(S_{0})$, columnas etiquetadas por las filas de $S_{0}$. Todas las columnas de $T_{1}$ están matcheadas, pues si no se podría agregar alguna alguna fila de $S_{0}$ al matching.
				\item $S_{1} =$ filas etiquetadas por las columnas de $T_{1}$.
				\item $T_{2} = \Gamma(S_{1}) - T_{1}$, columnas etiquetadas por las filas de $S_{1}$. 
			\end{itemize}
	
			En general:
			\begin{center}
			\begin{itemize}
				\item $S_{i} =$ filas matcheadas con $T_{i}$.
				\item $T_{i+1} = \Gamma(S_{i}) - (T_{1} \cup T_{2} \cup \dotsc T_{i}$.
			\end{itemize}
			\end{center}
			
			Como el algoritmo para sin hayar matching, entonces $\forall i \; T_{i} \neq \emptyset $, produce un $S_{i}$  (i.e $S_{i} \neq \emptyset$).
			
			$\therefore$ La única forma de parar es en un \textit{k}, tal que $T_{k+1} = \emptyset$.
			
			\underline{Observaciones:}
			\begin{enumerate}
				\item $\lvert S_{j} \rvert = \lvert T_{j} \rvert$, pues $S_{j}$ son las filas matcheadas con $T_{j}$.
				\item $\Gamma(S_{0} \cup S_{1} \cup \dotsc S_{j}) = T_{1} \cup T_{2} \cup \dotsc T_{j+1}$
				
					\underline{Por inducción en \textit{j}:}
					\begin{itemize}
						\item Caso Base: $j = 0$ vale.
						\item Caso Inductivo: Supongamos que vale para \textit{j}, veamos para \textit{j+1}.
						
						\begin{eqnarray}
							\nonumber T_{1} \cup T_{2} \dotsc T_{j+2} &=& T_{1} \cup T_{2} \cup \dotsc T_{j+1} \cup \underbrace{(\Gamma(S_{i+1}) - (T_{1} \cup T_{2} \dotsc T_{j+1}))}_{T_{j+1}} \\
							\nonumber &=& T_{1} \cup T_{2} \cup \dotsc T_{j+1} \cup (\Gamma(S_{i+1}) \\
							\nonumber &=& \Gamma(S_{0} \cup S_{1} \cup \dotsc S_{j}) \cup \Gamma(S_{i+1}) \qquad \text{Por H.I} \\
							\nonumber &=& \Gamma(S_{0} \cup S_{1} \cup \dotsc S_{j} \cup S_{j+1}) 
						\end{eqnarray}
					\end{itemize}
					
				\item Por construcción, los $S_{i}$ y $T_{i}$ son todos distintos.
				
					Sea $S = S_{0} \cup S_{1} \cup \dotsc S_{k}$
					
					\begin{eqnarray}
						\nonumber \lvert \Gamma(S) \rvert &=& \lvert \Gamma(S_{0} \cup S_{1} \cup \dotsc S_{k}) \rvert \\
						\nonumber &=& \lvert T_{1} \cup T_{2} \cup \dotsc \underbrace{T_{k+1}}_{= \; \emptyset} \rvert \qquad \; \text{Por 3} \\
						\nonumber &=& \lvert T_{1} \cup T_{2} \cup \dotsc T_{k} \rvert \\
						\nonumber &=& \lvert T_{1} \rvert + \lvert T_{2} \rvert + \dotsc \lvert T_{k} \rvert \qquad \text{Por 2} \\
						\nonumber &=& \lvert S_{1} \rvert + \lvert S_{2} \rvert + \dotsc \lvert S_{k} \rvert \qquad \text{Por 1} \\
						\nonumber &=& \lvert S \rvert - \lvert S_{0} \rvert \qquad \qquad \qquad \; \; \text{Por 2} \\
						\nonumber &<& \lvert S \rvert \qquad \text{Absurdo!}
					\end{eqnarray}
			\end{enumerate}
			
			
	\section{Teorema del matrimonio}
	
	
	\section{Si G es bipartito $\Rightarrow \chi '(G) = \Delta $}

	
	\section{Teorema cota de Hamming}
	
	
	\section{Sea H una matriz de chequeo de un código C, pruebe que:}
	
		\subsection{$\delta (C) =$ mínimo número de columnas linealmente dependientes de H}
		
		\subsection{Si H no tiene la columna cero ni columnas respetidas $\Rightarrow$ C corrige al menos un error}


	\section{Sea C un código cíclico de dimensión \textit{k} y longitud \textit{n} y sea $g(x)$ su polinomio generador, probar que:}
	
		\subsection{C está formado por los múltiplos de $g(x)$ de grado menor a \textit{n}}
		
		\subsection{El grado de $g(x)$ es $n-k$}
		
		\subsection{$g(x)$ divide a $1+x^{n}$}
		
		

\chapter{Parte C}

	\section{4-COLOR $\leq_{\textit{p}}$ SAT}
	
	\section{3-SAT es NP-Completo}
	
	\section{3-COLOR es NP-Completo}
	
	
\begin{thebibliography}{X}
\bibitem{Dan} \textsc{Curto Agustín },
<<Matemática Discreta II, apuntes de clase>>,
\textit{FaMAF, UNC}.
\bibitem{Baz} \textsc{Maximiliano Illbele},
<<Resumen de Discreta II, 16 de agosto de 2012>>,
\textit{FaMAF, UNC}.
\end{thebibliography}

\end{document}
