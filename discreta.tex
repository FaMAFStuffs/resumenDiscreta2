\documentclass[12pt,a4paper]{report}
\usepackage[utf8]{inputenc}
\usepackage[spanish]{babel}
\usepackage{amsmath}
\usepackage{amsfonts}
\usepackage{amssymb}
\usepackage{lmodern}
\usepackage{amsmath}
\usepackage[left=2cm,right=2cm,top=2cm,bottom=2cm]{geometry}
\usepackage{graphicx}
\author{Agustin Curto, agucurto95@gmail.com}
\title{Resumen de teoremas para el final \\ de Matemática Discreta  II}
\date{2016}

\begin{document}
\maketitle
\tableofcontents


\chapter{Parte A}

	\section{La complejidad de EDMONS-KARP}
	
	
	\section{Las distancias de Edmonds-Karp no disminuyen en pasos sucesivos}

	
	\section{La complejidad de DINIC}

	
	\section{La complejidad de WAVE}

	
	\section{La distancia entre NA sucesivos aumenta}
	


\chapter{Parte B}

	\section{2-COLOR es polinomial}
	
	
	\section{Probar que:}
	
		\subsection{El valor de todo flujo es menor o igual que la capacidad de todo corte.}
		
		\subsection{Si el valor de un flujo es igual a la capacidad de un corte entonces el flujo es maximal y el corte minimal.}
		
		\subsection{Si un flujo es maximal entonces existe un corte con capacidad igual al valor del flujo.}
	
	
	\section{Complejidad del Hungaro es $\mathcal{O}(n^{4})$}
	
	
	\section{Teorema de Hall}
	
	
	\section{Teorema del matrimonio}
	
	
	\section{Si G es bipartito $\Rightarrow \chi '(G) = \Delta $}

	
	\section{Teorema cota de Hamming}
	
	
	\section{Sea H una matriz de chequeo de un código C, pruebe que:}
	
		\subsection{$\delta (C) =$ mínimo número de columnas linealmente dependientes de H}
		
		\subsection{Si H no tiene la columna cero ni columnas respetidas $\Rightarrow$ C corrige al menos un error}


	\section{Sea C un código cíclico de dimensión \textit{k} y longitud \textit{n} y sea $g(x)$ su polinomio generador, probar que:}
	
		\subsection{C está formado por los múltiplos de $g(x)$ de grado menor a \textit{n}}
		
		\subsection{El grado de $g(x)$ es $n-k$}
		
		\subsection{$g(x)$ divide a $1+x^{n}$}
		
		

\chapter{Parte C}

	\section{4-COLOR $\leq_{\textit{p}}$ SAT}
	
	\section{3-SAT es NP-Completo}
	
	\section{3-COLOR es NP-Completo}
	
	
\begin{thebibliography}{X}
\bibitem{Dan} \textsc{Curto Agustín },
<<Matemática Discreta II, apuntes de clase>>,
\textit{FaMAF, UNC}.
\bibitem{Baz} \textsc{Maximiliano Illbele},
<<Resumen de Discreta II, 16 de agosto de 2012>>,
\textit{FaMAF, UNC}.
\end{thebibliography}

\end{document}